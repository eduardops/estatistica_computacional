

% UNIVERSIDADE FEDERAL DA PARAÍBA
% PROGRAMA DE PÓS-GRADUAÇÃO EM ECONOMIA

% Prof. Aléssio Almeida
% Prof. Hilton Ramalho



% ESTATÍSTICA COMPUTIONAL - AULA 2

%--------------------------------------

% EXEMPLO

% Classe do documento [opções]{classe}
\documentclass[12pt,a4paper]{article}


% Pacotes para codificação de caracteres especiais
\usepackage[T1]{fontenc}
\usepackage[utf8]{inputenc}

% Pacote para recuo de parágrafos
\usepackage{indentfirst}

% Pacote para margens do documento [opcoes de margens]
\usepackage[left=2.50cm, right=2.50cm, top=2.50cm, bottom=2.50cm]{geometry}

% Pacote para espaçamento entre linhas
\usepackage{setspace}

% Pacotes para símbolos matemáticos
\usepackage{amsmath}
\usepackage{amsfonts}
\usepackage{amssymb}

% Pacotes pra ajuste de texto
\usepackage{float}
\usepackage{fancyvrb}


% Comandos para ajustamento de texto

% Recuo do parágrafo:
\setlength{\parindent}{1.3cm}

% Controle do espaçamento entre um parágrafo e outro
% \onelineskip % alternativa
\setlength{\parskip}{0.25cm}  

% Espaçamento entre linhas (setspace package)

\singlespacing % Para um espaçamento simples
% \onehalfspacing %Para um espaçamento de 1,5
% \doublespacing %Para um espaçamento duplo



% Informações do documento
\title{Fichamento aula 2 -- Estatística Computacional} 
\author{Eduardo Paz Serafim}
\date{}



% Início do ambiente de texto
\begin{document}
	
% Criar cabeçalho com título e autor	
\maketitle	

\section{Ambientes para fórmulas matemáticas}
No \LaTeX\ existem alguns ambientes distintos para representação de fórmulas matemáticas, entre eles existem pequenas distinções que se adequam a diferentes formas de uso. A seguir está apresentada a estrutura básica, formas de alinhamento e alternativas comuns utilizadas.
\subsection{Estrutura básica}
O ambiente 'equation' pode ser utilizado com a seguinte sintaxe: 

\begin{BVerbatim}
\begin{equation}
	% conteúdo
\end{equation}
\end{BVerbatim}
\\
\\
Dentro do ambiente, o conteúdo inserido será interpretado como uma formula matemática, exemplos:

\begin{BVerbatim}
\begin{equation}
	x^2 + x + 0 = 0;
\end{equation}
\end{BVerbatim}
\\
\\
Resulta em:
\begin{equation}
	x^2 + x + 0 = 0;
\end{equation}
\\
Como se observa, o item por padrão é numerado. Caso não seja necessário a numeração, poderá ser realizado da seguinte forma:

\begin{BVerbatim}
\begin{equation*}
	\cos (2\theta) = \cos^2 \theta - \sin^2 \theta
\end{equation*}
\end{BVerbatim}
\\
\\
Resulta na equação não  numerada a seguir:
\begin{equation*}
	\cos (2\theta) = \cos^2 \theta - \sin^2 \theta
\end{equation*}

\subsection{Alinhamentos}
Ao utilizar diversas funções dentro de um mesmo ambiente pode ser necessário alinha-las para produzir um texto mais legível, para isso poderá ser utilizado o ambiente 'align' juntamente com o caractere '\&' que define o ponto de alinhamento no texto, no exemplo abaixo, podemos verificar a diferença no alinhamento:

\begin{BVerbatim}
\begin{align}
	&2x - 5y =  8 \\ \nonumber
	&3x + 9y + 2z =  -12 
\end{align}

\begin{align}
	2x - 5y &=  8 \\ \nonumber
	3x + 9y + 2z &=  -12
\end{align}
\end{BVerbatim}
\\
\\
A primeira equação foi alinhada pelo início, e a segunda pela posição do sinal de igualdade:
\begin{align}
	&2x - 5y =  8 \\ \nonumber
	&3x + 9y + 2z =  -12 
\end{align}
\begin{align}
	2x - 5y &=  8 \\ \nonumber
	3x + 9y + 2z &=  -12
\end{align}

O ambiente 'align' também é útil para apresentar várias formulas distribuídas ao longo da mesma linha, como no exemplo a seguir:

\begin{BVerbatim}
\begin{align*}
	x&=y           &  w &=z              &  a&=b+c\\
	2x&=-y         &  3w&=\frac{1}{2}z   &  a&=b\\
	-4 + 5x&=2+y   &  w+2&=-1+w          &  ab&=cb
\end{align*}
\end{BVerbatim}
\\
\\
Resultando em: 
\begin{align*}
	x&=y           &  w &=z              &  a&=b+c\\
	2x&=-y         &  3w&=\frac{1}{2}z   &  a&=b\\
	-4 + 5x&=2+y   &  w+2&=-1+w          &  ab&=cb
\end{align*}



\subsection{Alternativas}

As equações podem ser integradas a uma linha utilizando a o símbolo \textbf{cifrão}, vejamos o exemplo a seguir:

\begin{BVerbatim}
Temos que: $\pi \approx 3.1415 $.
\end{BVerbatim}
\\
\\
Resultando em: 

Temos que: $\pi \approx 3.1415 $.
\\
\\
De forma similar, poderíamos dar destaque utilizando \textbf{cifrão duplo}:


\begin{BVerbatim}
Temos que: $$\pi \approx 3.1415 $$.
\end{BVerbatim}
\\
\\
Resultando em: 

Temos que: $$\pi \approx 3.1415 $$.
\\
\\

\section{Representação de estruturas básicas}
\subsection{Exponenciação}

A sintaxe de exponenciação é dada pelo sinal \^ na seguinte forma \textbf{num \^\ expoente}, por exemplo:

\begin{BVerbatim}
$$ a = \pi . r^2 $$ 
\end{BVerbatim}
\\
\\
Resultando em: 

$$ a = \pi . r^2 $$ 
\\
\\

\subsection{Subscrito}

A sintaxe utilizada para subscrever é dada pelo sinal \_ na seguinte forma \textbf{num \_\ subscrito}, por exemplo:

\begin{BVerbatim}
$$ a_1 + a_2 = a_3 $$ 
\end{BVerbatim}
\\
\\
Resultando em: 

$$ a_1 + a_2 = a_3 $$ 
\\
\\

\subsection{Fração}

As frações podem ser utilizadas a partir do comando $\backslash frac\{num\}\{den\} $, por exemplo:

\begin{BVerbatim}
$$ a_1 = \frac{a_2}{a_3} $$ 
\end{BVerbatim}
\\
\\
Resultando em: 

$$ a_1 = \frac{a_2}{a_3} $$ 
\\
\\

\subsection{Raízes}

Raízes são utilizadas com o comando $ \backslash sqrt[root]\{arg\} $, por exemplo:

\begin{BVerbatim}
$$ \sqrt[3]{64} = 4 $$ 
\end{BVerbatim}
\\
\\
Resultando em: 

$$ \sqrt[3]{64} = 4 $$ 
\\
\\

\section{Tabelas}
\subsection{Estrutura básica}
\subsection{Alinhamentos e Bordas}
\subsection{Titulo e Fonte}

	
\end{document}