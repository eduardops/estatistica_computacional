

% UNIVERSIDADE FEDERAL DA PARAÍBA
% PROGRAMA DE PÓS-GRADUAÇÃO EM ECONOMIA

% Prof. Aléssio Almeida
% Prof. Hilton Ramalho



% ESTATÍSTICA COMPUTIONAL - AULA 1

%--------------------------------------

% EXEMPLO

% Classe do documento [opções]{classe}
\documentclass[12pt,a4paper]{article}


% Pacotes para codificação de caracteres especiais
\usepackage[T1]{fontenc}
\usepackage[utf8]{inputenc}

% Pacote para recuo de parágrafos
\usepackage{indentfirst}

% Pacote para margens do documento [opcoes de margens]
\usepackage[left=2.50cm, right=2.50cm, top=2.50cm, bottom=2.50cm]{geometry}

% Pacote para espaçamento entre linhas
\usepackage{setspace}

% Pacotes para símbolos matemáticos
\usepackage{amsmath}
\usepackage{amsfonts}
\usepackage{amssymb}

\usepackage{float}


% Comandos para ajustamento de texto

% Recuo do parágrafo:
\setlength{\parindent}{1.3cm}

% Controle do espaçamento entre um parágrafo e outro
% \onelineskip % alternativa
\setlength{\parskip}{0.25cm}  

% Espaçamento entre linhas (setspace package)

\singlespacing % Para um espaçamento simples
% \onehalfspacing %Para um espaçamento de 1,5
% \doublespacing %Para um espaçamento duplo



% Informações do documento
\title{Exercícios aula 2 -- Estatística Computacional} 
\author{Eduardo Paz Serafim}
\date{}



% Início do ambiente de texto
\begin{document}
	
% Criar cabeçalho com título e autor	
\maketitle	

1. 
$$ x_1, x_2 = \dfrac{ -\beta \pm \sqrt{\alpha^2 -4 \cdot \alpha \omega \cdot \gamma}}{2 \alpha \omega}, \ (\alpha^2 -4 \alpha \omega \gamma) > 0. $$


2.
\begin{align}
         x^2 + 2x - 15 &= 0,  \\
\Rightarrow (x+5)(x-3) &= 0, \nonumber  \\
         \Rightarrow x &= -5,\ 3.
\end{align}


3. 
\begin{equation*}
\sin 30^\circ = \frac{1}{2} = \frac{1}{\sqrt{3}}\sin 60^\circ = \cos (\pi / 2).
\end{equation*}

4.
\begin{equation*}
\arccos x = \int_x^1 \frac{\mathrm{d}u}{\sqrt{1-u^2}}.
\end{equation*}

5.
\begin{equation*}
n^\mathrm{th}, \qquad 1^\mathrm{st}, \qquad 2^\mathrm{nd}.
\end{equation*}

6.
\begin{equation*}
\begin{pmatrix}
F[1,1] & \cdots & F[1,m] \\
\vdots & \ddots & \vdots \\
F[n,1] & \cdots & F[n,m] 
\end{pmatrix}
\end{equation*}


7.
\begin{equation*}
\alpha = \frac{e^2}{2h_{\epsilon_0c}} \approx \frac{1}{137}, \ k=1.38 \times 10^{-23}\ \mathrm{JK}{-1}.
\end{equation*}

8.
\begin{align*}
f(x) &= \frac{a_0}{2} + \sum_{n=1}^{\infty}(a_n \cos nx + b_n \sin nx), \\
 a_n &= \frac{1}{\pi} \int_{-\pi}^{\pi} f(x) \cos nx \mathrm{d}x, n = 1,2,\dots \\
 b_n &= \frac{1}{\pi} \int_{-\pi}^{\pi} f(x) \sin nx \mathrm{d}x, n = 1,2,\dots
\end{align*}

	
	
Exercício 2.2

\begin{table}[H]
	\caption{Campos de formação do Curso de Economia}
	\centering
	\begin{tabular}{ l | c c }
		\hline
		\textbf{Campos de formação} & \textbf{CH-DCN2007} & \textbf{CH-UFPB} \\ \cline{2-3} 
		
		\textbf{Básico}\\  
		I-Conteúdos de Formação Geral 					& 10\% &20\% \\
		II-Conteúdos de Formação Teórico-Quantitativo 	& 20\% & 34\% \\
		III-Conteúdos de Formação Histórica 			& 10\% & 10\% \\
		
		\textbf{Complementar}\\  
		IV-Metodologia e TCC 							& 10\% & 18\% \\
		
		\textbf{Obrigatório} 							& 50\% & 82\% \\ 
		V-Conteúdos Teóricos-Práticos(Flexível)			& 50\% & 18\% \\ \hline
		
		\textbf{Total} 									& 100\% & 100\% \\ \hline
	\end{tabular}
\end{table}

	
\end{document}